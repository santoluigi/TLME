\documentclass[11pt]{article}
\usepackage{geometry} 
\geometry{a4paper,top=3cm,bottom=3cm,left=2.5cm,right=2.5cm}   
\usepackage{multicol}
\usepackage[english, italian]{babel}
\usepackage{fancyhdr}
\usepackage{musicography}
\geometry{centering}
\usepackage{graphicx}


\pagestyle{fancy}                                 %serve ad inserire la linea sopra e il titolino
\rhead{\textsf{Lezione 4 "\textit{Sintesi Additiva e DFT di un suono reale}"}}
\renewcommand{\headrulewidth}{5pt} %grandezza della linea in alto
\renewcommand{\footrulewidth}{1pt}   % grandezza della linea in basso


\begin{document}
\begin{minipage}{0.55\linewidth}
\vspace{0.3cm}
{\large{\textbf{\textsf{Gabriele Petrillo}}}}\\\end{minipage}

\vspace{0.3cm}
\begin{minipage}{0.95\linewidth}
\begin{center}
{\huge{\textbf{\textsf{Sintesi Additiva e DFT di un suono reale}}}} \\
\end{center}
\end{minipage}
\vspace*{0.2cm}


%=========ABSTRACT=======================
\begin{center}
\begin{minipage}[c]{14cm}
\begin{textit}

Ciò che segue è stato sviluppato a partire dalla dispensa di composizione del 17/02/2020 del prof. N.Bernardini presso il conservatorio S.Cecilia di Roma che possono essere consultati presso https://git.smerm.org/SMERM/TR-2019-2020/src/branch/master/COME-02/A.A.2019-2020/20200224. Questa dispensa tratta della sintesi additiva, la struttura delle forme d'onda canoniche e la DFT di un suono reale in Octave.

\end{textit}
\end{minipage}
\end{center}
\vspace*{0.2cm}

%=========ARTICOLO========================

\begin{multicols*}{2}
\parskip=0pt

\textbf{\textsf {DFT in Octave}}\\

\noindent Vediamo come realizzare una DFT (\textit{Discrete Fourier Transform}) di un suono reale in Octave.

\begin{center}
\begin{minipage}[c]{6.2cm}
\begin{sffamily}
\scriptsize

[y fc]=audioread("trumpet\_wav.wav");\\
y = y'; \\

nsamples=1000\\

dur=length(y)/fc;\\
sinc=$1/$fc;\\
t=[0:sinc:dur-sinc];\\
binsize=fc/nsamples;\\
f=[-fc/2 : binsize : fc/2 - binsize];\\
yfrag = y(5000 : 5000 + nsamples - 1);\\
tfrag = t(5000 : 5000 + nsamples - 1);\\

mydft = zeros(1, nsamples);\\

for k = 1 : length(f)\\
  anal = e.\wedge$(-i * 2 * pi * f(k) * tfrag)$;\\
  $inter = anal  * yfrag;\\
  temp = sum(inter);\\
  mydft(1,k) = temp;\\
end$\\

mag = abs(mydft)/nsamples;\\

figure(1)\\
stem(f, mag*2)\\
axis([0 7000])\\

\end{sffamily}
\end{minipage}
\end{center}\\

Come abbiamo già visto nelle lezioni precedenti l'opcode \textit {audioread} importa in una tabella un file audio, in questo caso trumpet\_wav.wav che si trova nel percorso di Octave. I campioni sono importati nell'array verticale y, tuttavia, per i nostri calcoli, è necessario avere i campioni disposti in un array orizzontale, per ottenere l'inverso di un array è necessario inserire un apice singolo accanto alla variabile dell'array, quindi: 

\begin{center}
\begin{minipage}[c]{6.2cm}
\begin{sffamily}
\scriptsize

y = y'; \\

\end{sffamily}
\end{minipage}
\end{center}\\

In questo modo abbiamo creato un array \textit{y} orizzontale.
A questo punto dobbiamo decidere il numero di campioni da analizzare, in questo caso 1000, quindi:
 
 \begin{center}
\begin{minipage}[c]{6.2cm}
\begin{sffamily}
\scriptsize

nsamples=1000\\

\end{sffamily}
\end{minipage}
\end{center}\\

Dobbiamo quindi determinare la durata in secondi del frammento \textit{dur} ed il suo periodo \textit{sinc}:

 \begin{center}
\begin{minipage}[c]{6.2cm}
\begin{sffamily}
\scriptsize

dur=length(y)/fc;\\
sinc=$1/$fc;\\

\end{sffamily}
\end{minipage}
\end{center}\\

Per costruire l'asse del tempo (asse x) useremo:

 \begin{center}
\begin{minipage}[c]{6.2cm}
\begin{sffamily}
\scriptsize

t=[0 : sinc : dur-sinc];\\

\end{sffamily}
\end{minipage}
\end{center}\\

La risoluzione della nostra DFT è definita dal parametro binszie che si definisce come: $binsize = \frac{fc}{nsamples}$, quindi:

 \begin{center}
\begin{minipage}[c]{6.2cm}
\begin{sffamily}
\scriptsize

binsize=fc/nsamples;\\

\end{sffamily}
\end{minipage}
\end{center}\\


Si noti che tra il vettore x e l’operatore di elevazione al quadrato è stato frapposto un punto. Il punto viene adoperato per effettuare operazioni su ogni singolo elemento del vettore che lo precede.

\section*{\centering\small{BIBLIOGRAFIA}}
•\textsc{\textsf {Bianchini, Riccardo and Cipriani, Alessandro}}, \emph{Il suono virtuale}, Contemponet 2001\\

\end{multicols*}

\end{document}